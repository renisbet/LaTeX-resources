\documentclass{article}
\usepackage[utf8]{inputenc}
\usepackage{graphicx} %you need this package if you want to include images!

\title{[title of experiment]}
\author{[name]}
\date{[date here]}

\begin{document}

\maketitle %you need this!

\begin{abstract}
    [put your abstract here]\cite{nameofreference}. Feel free to copy and paste this into your own document and then edit it.
\end{abstract}

\section{Theoretical background}
This is some text.\\ %the \\ makes a new line
This is a new line.

This is a new paragraph.\\ %leaving an empty line makes a new paragraph
Math mode: $x+y=1$ %this does maths in-line.

Equation envelopes:
\begin{equation}
    \int_{-\infty}^{\infty}x^2e^{-x^2}=\frac{\sqrt{\pi}}{2} %when you write in equation envelope you don't need to do math mode separately - it's automatic! it also numbers your equations for you. if you don't want it numbered, do \begin{equation*} with the *
\end{equation}

\section{Experimental Method}
\begin{figure}[h]%the 'h' makes sure the image goes here not anywhere else, otherwise LaTeX will decide where is best to put it. If it's still not going where you want, [h!] is overriding LaTeX's figure positioning rules so that will almost certainly work.
    \centering
    \includegraphics[width=85mm]{kitten.jpg}
    \caption{This is a kitten}
    \label{fig:kitten}
\end{figure}

\section{Results and Analysis}
Here's a table.
\begin{figure}[h]
    \centering
    \begin{tabular}{|p{2cm}|p{6cm}|}
    \hline
    \multicolumn{2}{|c|}{\textbf{Resistances}}\\
    \hline
    Symbol & Blood vessel \\
    \hline
    $ 1 $ & content \\
    $ 2 $ & content \\
    $ 3 $ &  content \\
    $ 4 $ & content \\
    $ 5 $ & content \\
    $ 6 $ & content \\
    $ 7 $ & content \\
    $ 8 $ & content \\
    $ 9 $ & content \\
    $ 10 $ & content \\
    $ 11 $ & content\\
    \hline
    \end{tabular}
    \caption{This is a table as a figure. You can also have tables outside figures, just omit the begin figure envelope.}
    \label{fig:my_label}
\end{figure}

\section{Discussion}
\subsection{This is a subsection}
\subsubsection{Common errors} %this is the max number of subsubsections you can do.
Common errors that come up in LaTeX: the yellow warning triangles. "Underfull hbox" usually means you've done a new line with two slashes and also a gap for a paragraph - there's no need to make a new line as well as a paragraph but ultimately it makes no difference.\\ %try putting an empty line under this
Overfull hbox usually means you've made a line too long in like an equation or something - it'll flow into the margins. Fix this by moving stuff onto a new line manually.


\section{Conclusion}

\begin{thebibliography}{9}
\bibitem{nameofreference}This Is A Reference.

\end{thebibliography}
\end{document}
